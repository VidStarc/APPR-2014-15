\documentclass[11pt,a4paper]{article}

\usepackage[slovene]{babel}
\usepackage[utf8x]{inputenc}
\usepackage{graphicx}

\pagestyle{plain}

\begin{document}
\title{Poročilo pri predmetu \\
Analiza podatkov s programom R}
\author{Vid Starc}
\maketitle

\section{Izbira teme}
Izbral sem si temo Število živine v Sloveniji in sicer bom predstavil število goveda in prašičev v različnih letih. Podatki so razdeljeni na vzhodno in zahodno Slovenijo. Poleg tega pa bom to število primerjal z številom goveda in prašičev po različnih državah Evrope. Podatke bom črpal iz statističnega urada republike Slovenije in iz uradne spletne strani Eurostat.

\section{Obdelava, uvoz in čiščenje podatkov}
Uvozil sem tabelo stevilo goveda in pri tem dodal še urejenostno spremenljivko ocena (primerjal sem slovensko povprečje in povprečje v štiriintridesetih državah v Evropi, torej ali je število posamezne vrste slabo zastopano, solidno ali odlično). Dodal sem stolpec povprečje pri kateremu sem izračunal vse povprečne vrednosti. V excelu sem dodal stolpec povprečje EU (ki obsega 34 držav v Evropi). Pri slednjem sem podatke črpal iz tabele, ki je na spletni strani Eurostat, seveda sem te povprečne vrednosti izračunal v Excelu. Tako pripravljeno tabelo (obsega 11 stolpcev in 69 vrstic) sem uvozil v program R, pri tem navedel še vse potrebne spremenljivke, imel pa sem težave z odpravljanjem znakov X v glavi tabele in pa znakov NA v tabeli.
Uvozil sem tabelo stevilo prašičev iz datoteke stevilo-prasisev.csv, ki sem jo našel na spletni strani SURS in nato še tabelo v obliki html datokeke stevilo.ovac, ki sem ji tudi dobil na SURS-u.
S pomočjo funkcije barplot sem narisal šest grafov in sicer število goveda v treh različnih letih, število prašičev v treh različnih regijah, pri čemer sem pri enemu dodal še legendo. Narisal sem še 3D tortni diagram ter vse skupaj shranil v dadoteko grafi.pdf.

\section{Analiza in vizualizacija podatkov}

\includegraphics{../slike/povprecna_druzina.pdf}

\section{Napredna analiza podatkov}

\includegraphics{../slike/naselja.pdf}

\end{document}
