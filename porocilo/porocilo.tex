\documentclass[11pt,a4paper]{article}

\usepackage[slovene]{babel}
\usepackage[utf8x]{inputenc}
\usepackage{graphicx}
\usepackage{url}
\usepackage{pdfpages}

\pagestyle{plain}

\begin{document}


\begin{titlepage}
\newcommand{\HRule}{\rule{\linewidth}{0.5mm}}
\center

\textsc{\LARGE Fakulteta za matematiko in fiziko}\\[3 cm]
\textsc{\Large Poročilo pri predmetu}\\[0.5cm]
\textsc{\large Analiza podatkov s programom R}\\[2 cm]
\HRule \\[0.4cm]
{ \huge \bfseries Število živine}\\[0.4cm] 
\HRule \\[6 cm]


\begin{minipage}{0.4\textwidth}
\begin{flushleft} \large
\emph{Avtor:}\\
Vid \textsc{Starc}
\end{flushleft}
\end{minipage}
~
\begin{minipage}{0.4\textwidth}
\begin{flushright} \large
\emph{Mentor:} \\
Dr. Janoš \textsc{Vidali}
\end{flushright}
\end{minipage}\\[2 cm]

{\large \today}\\[3cm] 


\end{titlepage}


\section{Izbira teme}
Izbral sem si temo Število živine v Sloveniji in sicer bom predstavil število goveda, prašičev in ovac v različnih letih. Podatki so razdeljeni na Vzhodno in Zahodno Slovenijo. Poleg tega pa bom to število primerjal s številom goveda in prašičev po različnih državah Evrope. Podatke bom črpal iz statističnega urada republike Slovenije in iz uradne spletne strani \verb|Eurostat|:
\begin{enumerate}
\item \url{http://pxweb.stat.si/pxweb/Database/Okolje/15_kmetijstvo_ribistvo/05_zivinoreja/01_15174_stevilo_zivine/01_15174_stevilo_zivine.asp}

\item \url{http://appsso.eurostat.ec.europa.eu/nui/show.do?dataset=apro_mt_lscatl&lang=en}

\item \url{http://appsso.eurostat.ec.europa.eu/nui/show.do?dataset=apro_mt_lspig&lang=en}
\end{enumerate}

Namen projekta je spoznati vsa potrebna orodja v programu R, ki služijo za kakovostno izdelavo projekta. Pri tem se bomo osredotočili na analizo podatkov, jih zapisovali v tabelo, risali podatkom primerne grafe in zemljevide.

\section{Obdelava, uvoz in čiščenje podatkov}
Uvozil sem tabelo \verb|število goveda| in pri tem dodal še urejenostno spremenljivko primerjava med letom 2013 in povprečjem (torej če je število goveda večje ali manjše). Dodal sem stolpec povprečje pri kateremu sem izračunal vse povprečne vrednosti ter stolpec povprečje EU (ki obsega 34 držav v Evropi). Pri slednjem sem podatke črpal iz tabele, ki je na spletni strani \verb|Eurostat|, seveda sem te povprečne vrednosti izračunal v Excelu. Tako pripravljeno tabelo (obsega 11 stolpcev in 69 vrstic) sem uvozil v program R, pri tem navedel še vse potrebne spremenljivke, imel pa sem težave z odpravljanjem znakov X v glavi tabele in pa znakov NA v tabeli.
Uvozil sem tabelo \verb|število prašičev| iz datoteke \verb|stevilo-prasisev.csv|, ki sem jo našel na spletni strani \verb|SURS| in nato še tabelo v obliki \textit{.htm} datokeke \verb|stevilo.ovac|, ki sem jo tudi dobil na \verb|SURS-u|.
S pomočjo funkcije barplot sem narisal šest grafov in sicer število goveda v treh različnih letih, število prašičev v treh različnih regijah in pri usakemu dodal še primerno legendo. Narisal sem še 3D tortni diagram ter vse skupaj shranil v dadoteke \verb|grafX.pdf|, kjer je X={1,2,...7}. Vse grafe si lahko ogledate nanaslednjih straneh.


\includegraphics[width=\textwidth]{../slike/graf1.pdf}

Graf primerja število goveda v letu 2007 v Vzhodni in Zahodni Sloveniji.


\includegraphics[width=\textwidth]{../slike/graf2.pdf}

Graf primerja število goveda v letu 2008 v Vzhodni in Zahodni Sloveniji.

\includegraphics[width=\textwidth]{../slike/graf3.pdf}

Graf primerja število goveda v letu 2008 v Vzhodni in Zahodni Sloveniji.

\includegraphics[width=\textwidth]{../slike/graf4.pdf}

Graf prikazuje število prašičev v Sloveniji v letu 2007, pri čemer je število v tisočih.

\includegraphics[width=\textwidth]{../slike/graf5.pdf}

Graf prikazuje število prašičev v Zahodni Sloveniji v letu 2007, pri čemer je število v tisočih.

\includegraphics[width=\textwidth]{../slike/graf6.pdf}

Graf prikazuje število prašičev v Zahodni Sloveniji v letu 2007, pri čemer je število v tisočih.

\includegraphics[width=\textwidth]{../slike/graf7.pdf}

Tortni diagram prikazuje deleže števila ovac V Sloveniji.


\newpage
\section{Analiza in vizualizacija podatkov}
Uvozil sem zemljevid Slovenije po regijah, ter regije razdelil na tiste, ki spadajo v Zahodno in na tiste, ki spadajo v Vzhodno Slovenijo, saj mi tako narekujejo podatki. Nato sem narisal štiri zemljevide, kjer je prikazano skupno število goveda po Vzhodni in Zahodni Sloveniji v letih 2010-20013. Po jakosti barv zelene in modre se lahko vidi, da je to število največje v letu 2010, najmanjše pa v letu 2012. Dodal sem še oznake regij in pri tem spremenil nekatere nastavitve pri koordinatah, tako da je ime lažje berljivo. Funkcije preuredi nisem potreboval, saj sem potreboval takšen vrstni red kot je v tabeli in sem zato delal samo z indeksi. Na koncu sem ta zemljevid shranil še v datoteko \verb|slovenija.pdf|. Spodaj je omenjeni zemljevid.

\makebox[\textwidth][c]{
\includegraphics[width=1.2\textwidth]{../slike/slovenija.pdf}
}
Zemljevidi prikazujejo število goveda po Vzhodni in Zahodni Sloveniji v letih 2010-2013.
\newline
\newline
\noindent Nato sem uvozil še eno tabelo iz strani \verb|Eurostat|, in sicer tabelo \verb|stevilo.govedaEU|, ki prikazuje skupno število goveda po različnih državah Evropske Unije v letih 2007-2013. To tabelo sem potreboval, ker sem želel uvoziti še zemljevid Evrope in nanj prikazati podatke, katere zajema moja tematika. Na zemljevidu, ki se nahaja spodaj sem prikazal skupno število goveda v letu 2013. Pri tem sem dodal še imena držav, popravil imena in  koordinate imen, izpustil Makedonijo in Luksemburg (ker sta bili njuni imeni na zemljevidu neberljivi), ter Bosno in Hercegovino skrajsal na BIH, Nizozemsko na NLD, Makedonijo na MKD, Dansko na DNK in Slovenijo na SLO. Tako urejen zemljevid sem shranil v datoteko \verb|evropa.pdf|.


\makebox[\textwidth][c]{
\includegraphics[width=1.2\textwidth]{../slike/evropa.pdf}
}
Zemljevid prikazuje skupno število goveda po državah EU v letu 2013.

% \newpage
% \section{Napredna analiza podatkov}


\end{document}
